\documentclass{article}

%some useful packages
\usepackage{amsmath,amsthm,amsfonts,amssymb}  %math stuff
\usepackage{graphicx} %to embed images
\usepackage{enumitem} %fancy lists
\usepackage{fullpage} %smaller margins
\usepackage{color} %smaller margins

%some common macros.
\renewcommand{\P}{\mathbb P}
\newcommand{\Z}{\mathbb Z}
\newcommand{\E}{\mathbb E}
\newcommand{\R}{\mathbb R}
\newcommand{\N}{\mathbb N}
\newcommand{\eps}{\varepsilon}
\DeclareMathOperator{\Var}{Var}
\DeclareMathOperator{\Poi}{Poi}
\DeclareMathOperator{\Cov}{Cov}
\DeclareMathOperator{\Exp}{Exp}
\DeclareMathOperator{\Bin}{Bin}
\DeclareMathOperator{\Geom}{Geom}
\DeclareMathOperator{\Bernoulli}{Bernoulli}

\theoremstyle{definition}
\newtheorem{problem}{Problem}
\setlist[itemize,1]{nosep}
\setlist[enumerate,1]{nosep,label=(\alph*)}

\newenvironment{solution}
{\paragraph{\color{blue}Solution.}}{\medbreak}

\usepackage{fancyhdr}
\fancypagestyle{crfooter}{%
  \fancyhf{}
  \renewcommand\headrulewidth{0pt}
  \fancyfoot[R]{\copyright\ Omer Angel 2022, all rights reserved.}
}

%%%%%%%%%%%%%%%%%%%%%%%%%%%%%%%%%%%%%%%%%%%%%%%%%%%%%%%%%%%%%%%%%%
\begin{document}

\thispagestyle{crfooter}

\begin{center}
  {\LARGE \textbf{Stochastic Processes}} \\ \medskip
  {\Large Assignment 8, due 2022-04-01}
\end{center}

\hrule width \textwidth

\paragraph{Note:} Start each problem on a \textbf{new page}.% (Multiple pages for a single problem are fine if needed.)

%%%%%%%%%%%%%%%%%%%%%%%%%%%%%%%%%%%%%%%%%%%%%%%%%%%%%%%%%%%%%%%%%%

\begin{problem}
  Consider the Yule process: a pure birth chain, where the rate of jumping from $n$ to $n+1$ is $\lambda n$.
  Suppose $X_0=1$.
  \begin{enumerate}
  \item Write down the backward Kolmogorov equations for $P_{ij}(t)$.
  \item Use these to find $P_{11}(t)$.
  \item Use these to find $P_{12}(t)$.
  \end{enumerate}
\end{problem}

%%%%%%%%%%%%%%%%%%%%%%%%%%%%%%%%%%%%%%%%%%%%%%%%%%%%%%%%%%%%%%%%%%

\begin{problem}
  A factory has three machines. Each breaks down at rate 1.
  If any machine is broken, a repairman works to fix it, and fixes at rate $2$.
  Let $X_t$ be the number of operational machines at time $t$.
  What is the limit probability for having no working machine?
  What is the limit probability for having all machines working?
\end{problem}

%%%%%%%%%%%%%%%%%%%%%%%%%%%%%%%%%%%%%%%%%%%%%%%%%%%%%%%%%%%%%%%%%%

\begin{problem}(Ross: 6.32):
  Customers arrive at a two-server station in accordance with a Poisson process
  having rate $\lambda$.
  Upon arriving, they join a single queue. Whenever a server completes a service, the person first in line enters service. The service times of
  server $i$ are exponential with rate $\mu_i$, $i=1,2$, where $\,u_1 + \mu_2 > \lambda$.
  An arrival finding both servers free is equally likely to go to either one. Define an appropriate continuous-time Markov chain for this model, show it is time reversible, and find the limiting probabilities.
\end{problem}

%%%%%%%%%%%%%%%%%%%%%%%%%%%%%%%%%%%%%%%%%%%%%%%%%%%%%%%%%%%%%%%%%%

\begin{problem}(Ross: 6.33)
  Consider two M/M/1 queues with respective parameters $\lambda_i,\mu_i$ for $i = 1,2$.
  Suppose they share a common waiting room that can hold at most three customers. 
  That is, whenever an arrival finds her server busy and three customers in the
  waiting room, she goes away.
  Find the limiting probability that there will be $n$ queue 1 customers and $m$ queue 2 customers in the system.
\end{problem}

%%%%%%%%%%%%%%%%%%%%%%%%%%%%%%%%%%%%%%%%%%%%%%%%%%%%%%%%%%%%%%%%%%

\begin{problem}
  A factory has $N$ identical machines.
  When a machine breaks down, its operator immediately begins to repair it.
  Each machine breaks down at rate $\mu$, and each repair independently takes an exponential time of rate $\lambda$.
  Let $X(t)$ denote the number of machines that are working at time $t$.
  This defines a birth and death process.
  \begin{enumerate}
  \item Determine the birth and death rates.
  \item Determine the limiting probabilities. (It is a certain binomial distribution.)
  \item Suppose that $N=50$, $\lambda = 10$, $\mu = 1$.
    What is the average number of machines that are operating, in the long run?
  \end{enumerate}
\end{problem}

%%%%%%%%%%%%%%%%%%%%%%%%%%%%%%%%%%%%%%%%%%%%%%%%%%%%%%%%%%%%%%%%%%

\hrule width \textwidth

\paragraph{Extra practice problems}
Do not hand these in. (Feel free to ask for hints if stuck.)

Ross, chapter 6: problems 20,24,26,38,39,42

\paragraph{Read ahead}
We will continue Chapter 6 next week, towards limiting probabilities.

\end{document}
