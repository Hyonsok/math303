\documentclass{article}

%some useful packages
\usepackage{amsmath,amsthm,amsfonts,amssymb}  %math stuff
\usepackage{graphicx} %to embed images
\usepackage{enumitem} %fancy lists
\usepackage{fullpage} %smaller margins
\usepackage{color} %smaller margins

%some common macros.
\renewcommand{\P}{\mathbb P}
\newcommand{\Z}{\mathbb Z}
\newcommand{\E}{\mathbb E}
\newcommand{\R}{\mathbb R}
\newcommand{\N}{\mathbb N}
\newcommand{\eps}{\varepsilon}
\DeclareMathOperator{\Var}{Var}
\DeclareMathOperator{\Poi}{Poi}
\DeclareMathOperator{\Cov}{Cov}
\DeclareMathOperator{\Exp}{Exp}
\DeclareMathOperator{\Bin}{Bin}
\DeclareMathOperator{\Geom}{Geom}
\DeclareMathOperator{\Bernoulli}{Bernoulli}

\theoremstyle{definition}
\newtheorem{problem}{Problem}
\setlist[itemize,1]{nosep}
\setlist[enumerate,1]{nosep,label=(\alph*)}

\newenvironment{solution}
{\paragraph{\color{blue}Solution.}}{\medbreak}

\usepackage{fancyhdr}
\fancypagestyle{crfooter}{%
  \fancyhf{}
  \renewcommand\headrulewidth{0pt}
  \fancyfoot[R]{\copyright\ Omer Angel 2022, all rights reserved.}
}

%%%%%%%%%%%%%%%%%%%%%%%%%%%%%%%%%%%%%%%%%%%%%%%%%%%%%%%%%%%%%%%%%%
\begin{document}

\thispagestyle{crfooter}

\begin{center}
  {\LARGE \textbf{Stochastic Processes}} \\ \medskip
  {\Large Assignment 5, due 2022-03-04}
\end{center}

\hrule width \textwidth

\paragraph{Note:} Start each problem on a \textbf{new page}.% (Multiple pages for a single problem are fine if needed.)

%%%%%%%%%%%%%%%%%%%%%%%%%%%%%%%%%%%%%%%%%%%%%%%%%%%%%%%%%%%%%%%%%%

\begin{problem}
  Suppose $X$ is an exponential random variavle with unknown rate $\lambda$.
  \begin{enumerate}
    \item Find $a$ for which $P(a\le X \le 2a)$ is maxized.
    \item If $P(X \le 5) = \frac23$, find $P(X\ge 10)$, and find $\lambda$.
  \end{enumerate}
\end{problem}

%%%%%%%%%%%%%%%%%%%%%%%%%%%%%%%%%%%%%%%%%%%%%%%%%%%%%%%%%%%%%%%%%%

\begin{problem}
  Suppose $X,Y$ are independent exponential variables with rates $\lambda,\mu$ respectively.
  \begin{enumerate}
  \item Prove that $P(X\le Y) = \frac{\mu}{\lambda+\mu}$.
  \item Calculate the conditional distribution of $X$ given that $X\le Y$.
  \end{enumerate}
\end{problem}

%%%%%%%%%%%%%%%%%%%%%%%%%%%%%%%%%%%%%%%%%%%%%%%%%%%%%%%%%%%%%%%%%%

\begin{problem}
  Recall the bank example from class.
  Suppose service at teller A takes $\Exp(\lambda_a)$ time and service at teller B takes $\Exp(\lambda_b)$ for some $\lambda_a,\lambda_b$.
  Naine arrives at the bank when the two tellers are serving customers, but no others are in line.
  Show that the probability that Naine leaves the bank after both previous customers is $2\frac{\lambda_a\lambda_b}{(\lambda_a+\lambda_b)^2}$.

  Hint: use the previous problem.
\end{problem}

%%%%%%%%%%%%%%%%%%%%%%%%%%%%%%%%%%%%%%%%%%%%%%%%%%%%%%%%%%%%%%%%%%

\begin{problem}
  Oaine is having office hours and Paine and Qaine will show up.
  The arrival times are independent exponentials with rates $\lambda_P$ and $\lambda_Q$.
  After arriving they stay for exponential times with rates $\mu_P$ and $\mu_Q$ respectively (all independent).
  \begin{enumerate}
  \item What is the probability that Paine comes and leaves before Qaine arrives?
  \item What is the expected time for the last student to leave?
  \end{enumerate}
\end{problem}

%%%%%%%%%%%%%%%%%%%%%%%%%%%%%%%%%%%%%%%%%%%%%%%%%%%%%%%%%%%%%%%%%%

\begin{problem}
  For a Poisson process $N(t)$ with rate $\lambda$, find the following:
  \begin{enumerate}
  \item $P(N(t)=n | N(s)=m)$ for $m\le n$ and $s\le t$.
  \item $P(N(s)=m | N(t)=n)$ for $m\le n$ and $s\le t$.
  \end{enumerate}
\end{problem}

%%%%%%%%%%%%%%%%%%%%%%%%%%%%%%%%%%%%%%%%%%%%%%%%%%%%%%%%%%%%%%%%%%

\hrule width \textwidth

\paragraph{Extra practice problems}
Do not hand these in. (Feel free to ask for hints is stuck.)

Ross, chapter 5: problems 3,7,10,18,23,26,39.

\paragraph{Read ahead}
We will finish Chapter 5 next week (Sections 5.3 and a part of 5.4).

\end{document}
