\documentclass{article}

%some useful packages
\usepackage{amsmath,amsthm,amsfonts,amssymb}  %math stuff
\usepackage{graphicx} %to embed images
\usepackage{enumitem} %fancy lists
\usepackage{fullpage} %smaller margins
\usepackage{color} %smaller margins

%some common macros.
\renewcommand{\P}{\mathbb P}
\newcommand{\Z}{\mathbb Z}
\newcommand{\E}{\mathbb E}
\newcommand{\R}{\mathbb R}
\newcommand{\N}{\mathbb N}
\newcommand{\eps}{\varepsilon}
\DeclareMathOperator{\Var}{Var}
\DeclareMathOperator{\Poi}{Poi}
\DeclareMathOperator{\Cov}{Cov}
\DeclareMathOperator{\Exp}{Exp}
\DeclareMathOperator{\Bin}{Bin}
\DeclareMathOperator{\Geom}{Geom}
\DeclareMathOperator{\Bernoulli}{Bernoulli}

\theoremstyle{definition}
\newtheorem{problem}{Problem}
\setlist[itemize,1]{nosep}
\setlist[enumerate,1]{nosep,label=(\alph*)}

\newenvironment{solution}
{\paragraph{\color{blue}Solution.}}{\medbreak}

\usepackage{fancyhdr}
\fancypagestyle{crfooter}{%
  \fancyhf{}
  \renewcommand\headrulewidth{0pt}
  \fancyfoot[R]{\copyright\ Omer Angel 2022, all rights reserved.}
}

%%%%%%%%%%%%%%%%%%%%%%%%%%%%%%%%%%%%%%%%%%%%%%%%%%%%%%%%%%%%%%%%%%
\begin{document}

\thispagestyle{crfooter}

\begin{center}
  {\LARGE \textbf{Stochastic Processes}} \\ \medskip
  {\Large Assignment 9, due 2022-04-08}
\end{center}

\hrule width \textwidth

\paragraph{Note:} Start each problem on a \textbf{new page}.% (Multiple pages for a single problem are fine if needed.)

%%%%%%%%%%%%%%%%%%%%%%%%%%%%%%%%%%%%%%%%%%%%%%%%%%%%%%%%%%%%%%%%%%

\begin{problem}
  We consider a model for a rumour (or a disease) spreading.
  The population has $N$ individuals.
  Initially only one knows the rumour.
  Each pair of individuals meet at rate 1. When that happens, if one knows the rumour they tell the other.
  \begin{enumerate}
  \item How many ways are there to go from $k$ individuals knowing the rumour to $k+1$?
  \item Let $X_t$ be the number of people who know the rumour at time $t$.
    Set this up as a continuous time Markov chain, and give the transition rates.
  \item Let $T_k$ be the time at which the chain reaches $k$. What is the distribution of $T_{k+1}-T_k$?
  \item Write an expression for $E(T_N)$: the time it takes for everyone to know the rumour.
    (You can leave this as a sum.)
  \end{enumerate}
\end{problem}

%%%%%%%%%%%%%%%%%%%%%%%%%%%%%%%%%%%%%%%%%%%%%%%%%%%%%%%%%%%%%%%%%%

\begin{problem}
  Consider the birth and death chain on $\{0,1,2,3,4\}$ with jump rates $\lambda_i = 4-i$ and $\mu_i = 2$.
  Suppose $X_0=2$.
  \begin{enumerate}
  \item  We wish to find the probability that the chain reaches state $4$ before reaching state 0.
    Let $u_i$ be the probability of reaching state 4 before state 0 if the chain starts at state $i$.
    Write equations for $u_i$ in terms of $u_{i-1}$ and $u_{i+1}$.
    (Hint: condition on the first jum of the chain.)
  \item Solve these equations to determine $u_i$ for every $i$.
  \item Next, we wish to find the expected time to reach state 4.
    Let $s_i$ be the expected time to reach state 4 if we start at state $i$.
    We can write that time a the time it takes to make the first jump, plus the time to reach 4 from whatever state the jump was to.
    Use this to write equations relating $s_i$ to $s_{i+1}$ and $s_{i-1}$.
  \item Solve these equations.
  \end{enumerate}
\end{problem}

%%%%%%%%%%%%%%%%%%%%%%%%%%%%%%%%%%%%%%%%%%%%%%%%%%%%%%%%%%%%%%%%%%

\begin{problem}
  Batman is chasing the joker.
  There are four places each could be, arranged in the corners of a square.
  Batman is moving from his corner to an adjacent corner at rate 1 (rate 1/2 to each).
  The Joker picks some rate $v$, and moves from his corner at rate $v$, in the same way.
  Find the expected time before they meet (are at the same corner).
  Hint: at each time they are either in opposite corners, nearby corners, or the same corner. Find transition probabilities between these.
\end{problem}

%%%%%%%%%%%%%%%%%%%%%%%%%%%%%%%%%%%%%%%%%%%%%%%%%%%%%%%%%%%%%%%%%%

\begin{problem}
  Batman is again chasing the Joker.
  This time the chase is in discrete time.
  In each step, Batman stays in his current corner with probability $\alpha$, and otherwise moves to a random nearby corner.
  The joker stays in place with probability $\beta$ and otherwise jumps to a nearby corner.
  \begin{enumerate}
  \item As in the previous problem, find the transition probabilities for the distance between Batman and the Joker.
  \item If they start at opposite corners, find the expected time to meet.
    Note: The case $\alpha=\beta=0$ is different!
  \end{enumerate}
\end{problem}

%%%%%%%%%%%%%%%%%%%%%%%%%%%%%%%%%%%%%%%%%%%%%%%%%%%%%%%%%%%%%%%%%%

\hrule width \textwidth

\paragraph{Read ahead}
Next week we will look at more examples of markov chains, and have a general review. Prepare questions you would like to see discussed in class.

\end{document}
