\documentclass{article}

%some useful packages
\usepackage{amsmath,amsthm,amsfonts,amssymb}  %math stuff
\usepackage{graphicx} %to embed images
\usepackage{enumitem} %fancy lists
\usepackage{fullpage} %smaller margins
\usepackage{color} %smaller margins

%some common macros.
\renewcommand{\P}{\mathbb P}
\newcommand{\Z}{\mathbb Z}
\newcommand{\E}{\mathbb E}
\newcommand{\R}{\mathbb R}
\newcommand{\N}{\mathbb N}
\newcommand{\eps}{\varepsilon}
\DeclareMathOperator{\Var}{Var}
\DeclareMathOperator{\Poi}{Poi}
\DeclareMathOperator{\Cov}{Cov}
\DeclareMathOperator{\Exp}{Exp}
\DeclareMathOperator{\Bin}{Bin}
\DeclareMathOperator{\Geom}{Geom}
\DeclareMathOperator{\Bernoulli}{Bernoulli}

\theoremstyle{definition}
\newtheorem{problem}{Problem}
\setlist[itemize,1]{nosep}
\setlist[enumerate,1]{nosep,label=(\alph*)}

\newenvironment{solution}
{\paragraph{\color{blue}Solution.}}{\medbreak}

\usepackage{fancyhdr}
\fancypagestyle{crfooter}{%
  \fancyhf{}
  \renewcommand\headrulewidth{0pt}
  \fancyfoot[R]{\copyright\ Omer Angel 2022, all rights reserved.}
}

%%%%%%%%%%%%%%%%%%%%%%%%%%%%%%%%%%%%%%%%%%%%%%%%%%%%%%%%%%%%%%%%%%
\begin{document}

\thispagestyle{crfooter}

\begin{center}
  {\LARGE \textbf{Stochastic Processes}} \\
  {\Large Assignment 4, due 2022-02-18}
\end{center}

\hrule width \textwidth \bigskip

\paragraph{Note:} Start each problem on a \textbf{new page}. (Multiple pages for a single problem are fine if needed.)

%%%%%%%%%%%%%%%%%%%%%%%%%%%%%%%%%%%%%%%%%%%%%%%%%%%%%%%%%%%%%%%%%%

\begin{problem}
  Is it possible for a branching process to be reversible?
  What can be said about $\xi$ in that case (The number of children of each individual is an independent copy of $\xi$.)
\end{problem}

%%%%%%%%%%%%%%%%%%%%%%%%%%%%%%%%%%%%%%%%%%%%%%%%%%%%%%%%%%%%%%%%%%

\begin{problem}
  Find the probability generating function $G(s) = \E s^X$ for the following distributions:
  \begin{enumerate}
  \item $X=\Bernoulli(p)$: here $P(X=1)=p$ and $P(X=0)=1-p$.
  \item $X=\Bin(n,p)$.
  \item $X=\Poi(\lambda)$.
  \item $X=\Geom(p)$, so $P(X=n) = p(1-p)^{n-1}$ for $n=1,2,\dots$.
  \end{enumerate}
\end{problem}

%%%%%%%%%%%%%%%%%%%%%%%%%%%%%%%%%%%%%%%%%%%%%%%%%%%%%%%%%%%%%%%%%%

\begin{problem}
  Find the probability generating function $G(s) = \E s^X$ for the following distributions:
  \begin{enumerate}
  \item $X=A+B$ where $A=\Bin(n,p)$ and $B=\Geom(p)$ are independent.
  \item Let $N=\Poi(\lambda)$ and $Y_1,Y_2,\dots$ be $\Geom(p)$ for some $p,\lambda$ and all are independent.
    Let $X = Y_1+\dots+Y_N$.
  \item Let $N=\Geom(q)$ and $Y_1,Y_2,\dots$ be $\Geom(p)$ for some $p,q$ and all are independent.
    Let $X = Y_1+\dots+Y_N$.
  \end{enumerate}
\end{problem}

%%%%%%%%%%%%%%%%%%%%%%%%%%%%%%%%%%%%%%%%%%%%%%%%%%%%%%%%%%%%%%%%%%

\begin{problem}
  Consider a branching process with offspring distribution $\xi$ with
  \begin{align*}
    P(\xi=0) &= \alpha & 
    P(\xi=1) &= \beta & 
    P(\xi=2) &= 0 & 
    P(\xi=3) &= 1-\alpha-\beta.
  \end{align*}
  \begin{enumerate}
  \item If $\alpha=\beta=1/3$, find the probability the process becomes extinct.
  \item If we start with $4$ individuals instead of 1, find the probability the process becomes extinct.
  \item If $\alpha,\beta$ are such that $\E(\xi)=2$, what $\alpha$ and $\beta$ minimize the probability of extinction (starting with 1 individual)? What $\alpha,\beta$ maximize that probability?
  \end{enumerate}
\end{problem}

%%%%%%%%%%%%%%%%%%%%%%%%%%%%%%%%%%%%%%%%%%%%%%%%%%%%%%%%%%%%%%%%%%

\begin{problem}
  Find transition probabilities on $S=\{0,1,2,\dots\}$ so that the stationary distribution is $\Poi(\lambda)$ and the only transitions that can hove non-zero probability are $P_{n,n-1}, P_{n,n+1}$, and $P_{n,n}$ for all $n$.
\end{problem}

%%%%%%%%%%%%%%%%%%%%%%%%%%%%%%%%%%%%%%%%%%%%%%%%%%%%%%%%%%%%%%%%%%

\paragraph{Extra practice problems}
Do not hand these in. (Feel free to ask for hints is stuck.)

Ross, chapter 4: problems 64,66,71,72,73,74.

\paragraph{Read ahead}
We will start Chapter 5 next week.

\end{document}
