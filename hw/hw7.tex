\documentclass{article}

%some useful packages
\usepackage{amsmath,amsthm,amsfonts,amssymb}  %math stuff
\usepackage{graphicx} %to embed images
\usepackage{enumitem} %fancy lists
\usepackage{fullpage} %smaller margins
\usepackage{color} %smaller margins

%some common macros.
\renewcommand{\P}{\mathbb P}
\newcommand{\Z}{\mathbb Z}
\newcommand{\E}{\mathbb E}
\newcommand{\R}{\mathbb R}
\newcommand{\N}{\mathbb N}
\newcommand{\eps}{\varepsilon}
\DeclareMathOperator{\Var}{Var}
\DeclareMathOperator{\Poi}{Poi}
\DeclareMathOperator{\Cov}{Cov}
\DeclareMathOperator{\Exp}{Exp}
\DeclareMathOperator{\Bin}{Bin}
\DeclareMathOperator{\Geom}{Geom}
\DeclareMathOperator{\Bernoulli}{Bernoulli}

\theoremstyle{definition}
\newtheorem{problem}{Problem}
\setlist[itemize,1]{nosep}
\setlist[enumerate,1]{nosep,label=(\alph*)}

\newenvironment{solution}
{\paragraph{\color{blue}Solution.}}{\medbreak}

\usepackage{fancyhdr}
\fancypagestyle{crfooter}{%
  \fancyhf{}
  \renewcommand\headrulewidth{0pt}
  \fancyfoot[R]{\copyright\ Omer Angel 2022, all rights reserved.}
}

%%%%%%%%%%%%%%%%%%%%%%%%%%%%%%%%%%%%%%%%%%%%%%%%%%%%%%%%%%%%%%%%%%
\begin{document}

\thispagestyle{crfooter}

\begin{center}
  {\LARGE \textbf{Stochastic Processes}} \\ \medskip
  {\Large Assignment 7, due 2022-03-25}
\end{center}

\hrule width \textwidth

\paragraph{Note:} Start each problem on a \textbf{new page}.% (Multiple pages for a single problem are fine if needed.)

%%%%%%%%%%%%%%%%%%%%%%%%%%%%%%%%%%%%%%%%%%%%%%%%%%%%%%%%%%%%%%%%%%

\begin{problem}
  Potential customers arrive at a queue at times of a Poisson process with rate $\lambda$.
  If a customer sees $n$ others in the queue, the join with probability $\alpha_n$, and give up and go home with probability $1-\alpha_n$.
  The service rate is always $\mu$.
  Set this up as a birth and death process and determine the birth and death rates.
\end{problem}

%%%%%%%%%%%%%%%%%%%%%%%%%%%%%%%%%%%%%%%%%%%%%%%%%%%%%%%%%%%%%%%%%%

\begin{problem}
  Consider two machines, both of which have an exponential lifetime with parameter $\lambda$.
  There is a single repairman can service machines at an exponential rate $\mu$.
  Let $X_t$ be the number of machines that are operational at time $t$.
  Write the Kolmogorov backward equations (you need not solve them).
\end{problem}

%%%%%%%%%%%%%%%%%%%%%%%%%%%%%%%%%%%%%%%%%%%%%%%%%%%%%%%%%%%%%%%%%%

\begin{problem}
  Consider a Yule process with parameters $\lambda_n = n\lambda$ and $\mu_n=0$.
  Start the process with $X_0=1$ individual.
  Let $T_n$ be the time at which the population reaches size $n$ (so $T_1=0$).
  \begin{enumerate}
  \item Explain why $T_{n+1}-T_n$ are independent exponential variables, and give their parameters.
  \item Find $E(T_n)$. (You do not need to simplify sums.)
  \end{enumerate}
\end{problem}

%%%%%%%%%%%%%%%%%%%%%%%%%%%%%%%%%%%%%%%%%%%%%%%%%%%%%%%%%%%%%%%%%%

\begin{problem}
  A continuous time Markov chain on states $\{a,b,c\}$ has jump rates
  \begin{align*}
    q_{ab} &= 1 &    q_{ba} &= 2 &    q_{ca} &= 2 \\
    q_{ac} &= 1 &    q_{bc} &= 1 &    q_{cb} &= 2.
  \end{align*}
  Suppose $X_0=a$. We wish to find the expected time to reach state $b$, denoted $T_b$.
  \begin{enumerate}
  \item What is the expected time for the first jump out of $a$?
  \item Once the chain leaves $a$, what is the distribution of the next state?
  \item Let $M_i$ be the expected time to reach $b$ if we start at state $i$. (So $M_b=0$). Use parts (a),(b) to write an equation for $M_a$ in terms of $M_b$ and $M_c$.
  \item Write a similar equation for $M_c$.
  \item Solve the equations to find $M$.
  \end{enumerate}
\end{problem}

%%%%%%%%%%%%%%%%%%%%%%%%%%%%%%%%%%%%%%%%%%%%%%%%%%%%%%%%%%%%%%%%%%

\begin{problem}[Ross, 6.2]
  A one-celled organism can be in one of two states: A or B.
  An individual in state A will change to state B at an exponential rate $\alpha$.
  An individual in state B divides into two new individuals of type A at an exponential rate $\beta$.
  Define an appropriate continuous-time Markov chain for a population of such organisms and determine the appropriate parameters for this model (the transition rates and jump probabilities.)
  (Hint: the state $(N_A,N_B)$ is the number of individuals of each type.)
\end{problem}

% \begin{problem}
%   A factory has $N$ identical machines.
%   When a machine breaks down, its operator immediately begins to repair it.
%   Each machine breaks down at rate $\mu$, and each repair independently takes an exponential time of rate $\lambda$.
%   Let $X(t)$ denote the number of machines that are working at time $t$.
%   This defines a birth and death process.
%   \begin{enumerate}
%   \item Determine the birth and death rates.
%   \item Determine the limiting probabilities. (It is a certain binomial distribution.)
%   \item Suppose that N=50, λ = 10, μ = 1. What is the average number of machines that are operating, in the long run?
%   \end{enumerate}
% \end{problem}

%%%%%%%%%%%%%%%%%%%%%%%%%%%%%%%%%%%%%%%%%%%%%%%%%%%%%%%%%%%%%%%%%%

\hrule width \textwidth

\paragraph{Extra practice problems}
Do not hand these in. (Feel free to ask for hints if stuck.)

Ross, chapter 6: problems 1,3,4,9,12.

\paragraph{Read ahead}
We will continue Chapter 6 next week, towards limiting probabilities.

\end{document}
